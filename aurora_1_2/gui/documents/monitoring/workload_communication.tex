\documentclass[a4paper,12pt]{article}
\begin{document}

\newcommand{\thisproj}{\bf GUI/Workload Communication}

\makeatletter                   % Make '@' accessible.
\pagestyle{myheadings}              % We do our own page headers.
\def\@oddhead{\bf Aurora - \thisproj \hfill (arsinger, yx)} 
\hbadness=10000                 % No "underfull hbox" messages.
\makeatother     

\section*{Overview}

The Workload generator is intended to provide sample data in order to test an
Aurora network. The user specifies desired attribute set for each external input
that they wish to test via a GUI interface. This information will then be 
communicated to the Workload generator, and stored in the catalog. Any changes 
to the values for the information generated will be pushed out the the Workload
generator as they are applied.

\section*{Information Handling}

In general, all information generated by the Workload generator will be treated
as an Aggregate Stream. Each Aggregate Stream comprises any number of streams.
The user will be able to specify the number of Streams within each Aggregate
Stream. 

All such streams are expected to have a rate of generation, a set
of fields, and total number of tuples to be generated. The set of fields to
be generated is the same for all streams within a single Aggregate Stream, as is
the total number of tuples to be generated. The values of the fields
themselves will vary. The manner in which they vary should be the same
across all streams in an Aggregate Stream.

To this end, the GUI will provide the user with the ability to create "Field
Sets," each of which is associated with an Aggregate Stream via a one to one
relationship. The Field Set will have several parameters in addition to the
actual attributes themselves. These parameters will be:

\begin{enumerate}
\item Unique ID
\item Port number (each associated with one box)
\item A set of Fields to be generated
\item Rate of generation
\begin{enumerate}
\item Variance of the rate (e.g. constant/periodic, or distribution)
\item Variables associated with the variance (e.g. mean, standard deviation,
etc.)
\end{enumerate}
\item Total number of tuples to generate (-1 means infinite)
\item Number of streams
\end{enumerate}

Each field within an Field Set will also have several parameters, as
follows:

\begin{enumerate}
\item Unique ID
\item Variance of the field
\begin{enumerate}
\item Kind of variance (random walk, constant, linear, timestamp, etc.)
\item Variables associated with the variance (e.g. mean, standard deviation,
etc.)
\end{enumerate}
\item Type (int, float, byte, etc.)
\end{enumerate}

Most of this information will be packaged and sent to the Workload generator via
a TCP or UDP stream as specified below.


\section*{Splitting Aggregate Streams}

Every input node to the system contains only streams of the same Aggregate
Stream. However, each Aggregate Stream may be split over several input nodes. 
For example, if an Aggregate Stream represents all of the soldiers in the field,
an operator may choose to deal with the first half of the soldiers separately 
from the second half, as they are part of different platoons. 

As the Aggregate Stream may split from the input port node, the user may choose
to create multiple arcs from the same input port if all of those sub-streams are
to have the same information carried across them, and are to be generated in the
same manner. However, if two groups of streams within an aggregate stream are to
be generated at different rates or with different parameters (e.g. simulate a
patient with a heart condition rather than one without), the user must specify
two separate input ports and set different Workload parameters on each.

\section*{Communication}

It is imperative that the GUI be able to send information to the Workload
generator during program operation so that an operator testing the system may
see how different values of rate or distribution values effect the system as a
whole. In addition, the operator should be able to start and stop the Workload
generator from within the GUI.

\subsection*{Method of Data Transfer}

Data will be sent from the GUI to the Workload generator over TCP or UDP, to be
decided, in order to facilitate quick transition in the generation of new data
coming into the system. This will allow users to change values, and see results
as soon as possible. The information itself will be sent on the level of the
input port nodes, sending one packet per type of Workload generation desired.
Until it receives another packet with the same port number, the Workload
generator will continue producing tuples in the specified data set.

\subsection*{Composition of Data}

The standard packet will have four major sections, each broken down into smaller
parts as information requires. A sample packet will look like:

|Rate length|Stream Information length|Tuple variance length|Rate variance
type|Rate variance|Rate variance distribution|Rate variance distribution 
parameters|Port|Number of streams|Number of fields|[Field Info Length|Field 
type|Field Variance type|Field Variance Params...|]

(Note that the last section may be repeated numerous times

The data will comprise several segments of data within the same packet. These
will be:

\begin{enumerate}
\item The lengths of each of the three following segments
\item Data concerning the Rate of production
\item Overall stream information
\item Tuple data variance
\end{enumerate}

All information will be sent as doubles to enable a standarized method of
writing and reading data on either end. In addition, all lengths will be
calculated in number of doubles (8 bytes) contained in each section.

The first segment of data will contain the lengths of the three other segments.
This should be three doubles, specifying the length of the Rate information,
stream information, and tuple variance, respectively.

The second section contains information about the desired rate and the variance
of the rate itself.

The third section is two static pieces of information- the port to connect to
and the number of different streams to be passed through that port. The port is
associated via a one to one correspondence with the top level Input Ports and is
generated by the GUI as a number between 5000 and 6000.

The last section contains all information necessary to generate random tuples
within each stream. The type and variance together signify one Field within each
tuple.

\end{document}
