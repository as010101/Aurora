\documentclass[a4paper,12pt]{article}
\begin{document}

\newcommand{\thisproj}{\bf GUI Workflow}

\makeatletter                   % Make '@' accessible.
\pagestyle{myheadings}              % We do our own page headers.
\def\@oddhead{\bf Aurora - \thisproj \hfill (arsinger, rly, yx)} 
\hbadness=10000                 % No "underfull hbox" messages.
\makeatother     

\section*{Overview}

The Workflow generator is intended to provide sample data in order to test an
Aurora network. The user specifies desired attribute set for each external input
that they wish to test. This information will then be communicated to the
Workflow generator. Any changes to the values for the information generated will
be pushed out the the Workflow generator as they are applied.

\section*{Information Handling}

In general, all information generated by the Workflow generator will be treated
as an Aggregate Stream. Each Aggregate Stream comprises any number of streams.
The user will be able to specify the number of Streams within each Aggregate
Stream. 

All such streams are expected to have a rate of generation, a set
of fields, and total number of tuples to be generated. The set of fields to
be generated is the same for all streams within a single Aggregate Stream, as is
the total number of tuples to be generated. The values of the fields
themselves will vary. The manner in which they vary should be the same
across all streams in an Aggregate Stream.

To this end, the GUI will provide the user with the ability to create "Field
Sets", each of which is associated with an Aggregate Stream via a one to one
relationship. The Field Set will have several parameters in addition to the
actual attributes themselves. These parameters will be:

\begin{enumerate}
\item Unique ID
\item A set of Fields to be generated
\item Rate of generation
\begin{enumerate}
\item Variance of the rate (e.g. constat/periodic, or distribution)
\item Variables associated with the variance (e.g. mean, standard deviation,
etc.)
\end{enumerate}
\item Total number of tuples to generate (-1 means infinite)
\item Number of streams
\end{enumerate}

Each field within an Field Set will also have several parameters, as
follows:

\begin{enumerate}
\item Unique ID
\item Variance of the field
\begin{enumerate}
\item Kind of variance (random walk, constant, linear, timestamp, etc.)
\item Variables associated with the variance (e.g. mean, standard deviation,
etc.)
\end{enumerate}
\item Type (int, float, byte, etc.)
\end{enumerate}

All of this information will be packaged and sent to the Workflow generator in a
manner to be determined.

\section*{Splitting Aggregate Streams}

Every input node to the system contains only streams of the same Aggregate
Stream. However, each Aggregate Stream may be split over several input nodes. 
For example, if an Aggregate Stream represents all of the soldiers in the field,
an operator may choose to deal with the first half of the soldiers separately 
from the second half, as they are part of different platoons. Therefore, the 
operator must have a GUI interface with which to specify where the various 
streams within an Aggregate Stream are to be sent.

This should be done by opening the Field Set properties and then indicating
blocks of streams within the Aggragate Stream and which input nodes they are to
be sent through. As this is also where the operator specifies the total number
of streams, this seems a logical place to insert this functionality.

\section*{Communication}

It is imperative that the GUI be able to send information to the Workflow
generator during program operation so that an operator testing the system may
see how different values of rate or distribution values effect the system as a
whole. In addition, the operator should be able to start and stop the Workflow
generator from within the GUI.

The manner in which this communication is to take place is not yet clear. Two
different options have been discussed thus far. The first is for the GUI to
push all information about Workflow generation to SleepyCat which will then push
it to the Workflow generator. The other is for the two components to communicate
directly via TCP/IP.

\end{document}
