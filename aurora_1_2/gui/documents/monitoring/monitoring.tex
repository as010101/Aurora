\documentclass[a4paper,12pt]{article}
\begin{document}

\newcommand{\thisproj}{\bf GUI Monitoring}

\makeatletter                   % Make '@' accessible.
\pagestyle{myheadings}              % We do our own page headers.
\def\@oddhead{\bf Aurora - \thisproj \hfill (arsinger)} 
\hbadness=10000                 % No "underfull hbox" messages.
\makeatother     

\section*{Overview}

The Monitoring Application is designed to allow a user of the Aurora system to
see any bottlenecks or overflows in an Aurora network. In order to achieve this,
users will be given the option of several different view options, each dependent
on the value to be measured. This data should be presented in a clear manner. In
addition, the user may have control over what values they wish to measure.

\section*{Method of Monitoring}

There are several possibilities with regards to what values can be measured and
how this information can be displayed to the user. The top candidates at present
are as follows:

\begin{enumerate}
\item What to measure
\begin{enumerate}
\item Latency between time of arrival at an arc and departure time
\item Queue length on an arc
\item Rate at which data reaches or leaves an arc
\item Length of time it takes a tuple to be computed within a box
\end{enumerate}
\item How to present system status
\begin{enumerate}
\item Color arcs (some code already in place to support)
\item Text label on arc
\item Tooltip on mouseover
\item Width of the arrow
\item Attach a QoS graph to the arc
\end{enumerate}
\end{enumerate}

\section*{Action Items}

The following items need to be accomplished:

\begin{itemize}
\item Retrieve data from C++ via SleepyCat
\begin{itemize}
\item Develop table schema
\item Communicate with whomever is storing said data
\end{itemize}
\item Research awt/swing
\begin{itemize}
\item for labeling arcs
\item for adding tooltips
\item for adding graph labels
\end{itemize}
\item Decisions regarding method
\begin{itemize}
\item Primary method
\item Pick which other methods to implement
\item Determine whether/how to allow user to change method used
\end{itemize}
\end{itemize}

\section*{Summer Plan}

In order to test monitoring, there must be data to monitor. The workflow
generator will provide this dynamically generated data. To this end, setting up
communication with the Workflow generator is the first priority, followed by
implementing monitoring capability.

The current plan for implementing Monitoring is as follows:
\begin{enumerate}
\item Initial GUI support for Monitoring (color arcs) [complete]
\item GUI end support for workflow generator [7/1]
\begin{enumerate}
\item Updating type/aggregate stream system
\item Implementing dialogs to set data values
\end{enumerate}
\item Communication with workflow generator (SleepyCat or TCP/IP) [7/10]
\item Communication with Scheduler via SleepyCat [7/14]
\item Connecting info from scheduler with several GUI methods [7/17]
\begin{enumerate}
\item Color arcs
\item Text label
\item Tooltip
\item Arrow width?
\end{enumerate}
\item Integrate with QoS graphs [7/26]
\item Revisions to UI [8/2]
\item User ability to alter how data is display and what data is recorded [8/9]
\end{enumerate}

\end{document}
