\documentclass[a4paper,12pt]{article}
\begin{document}

\newcommand{\thisproj}{\bf GUI LaTeX Handbook}

\makeatletter                   % Make '@' accessible.
\pagestyle{myheadings}              % We do our own page headers.
\def\@oddhead{\bf Aurora - \thisproj \hfill (arsinger)} 
\hbadness=10000                 % No "underfull hbox" messages.
\makeatother     

\section*{Making a LaTeX File}

The best method is to simply copy a previous LaTeX file rather than to start one
from scratch. See README\_LATEX.tex for an example.

With this, you will be all set to insert your own formatted text.

\section*{Special Tags}

In the README\_LATEX.tex file, there are several tags of note:

\begin{itemize}
\item newcommand: This is the subtitle of the header across each page
\item def @oddhead: This is what appears across the top of the page
\item section*\{\}: Section titles
\item begin\{enumerate\}: Start of a numbered list
\item end\{enumerate\}: End of a numbered list
\item begin\{itemize\}: Start of an unnumbered list
\item end\{itemize\}: End of an unnumbered list
\item item: A list item for either list
\end{itemize}

\section*{Normal Text:}

Simple type the text in. If you want to start a new paragraph, use two
returns to make it clear to a reader of the tex file that you have started a
new paragraph. LaTeX automatically formats for you.

\section*{Generating a *.ps file:}

You need to convert the *.tex file to a *.dvi file, and from the *.dvi file 
to a *.ps file. And you don't want too many intermediary files hanging around
afterwards, so we use the -c option. The -o "" option is to make sure we
output to a file rather than print.

Use:
\begin{enumerate}
\item texi2dvi4a2ps -c (filename).tex
\item dvips -o "" (filename).dvi
\end{enumerate}

This can also be achieved by writing:
texi2dvi4a2ps -c (filename).tex ; dvips -o "" (filename).dvi

As this is quite annoying to type, I have written a quick script:
/u/arsinger/bin/tex2ps
which takes just the first part of the name (not the .tex part) as a
variable. Also, it should be able to handle multiple files at a time.

So if you want to PSify:
my\_tex\_file.tex another\_tex.tex joe.tex
you would use:
/u/arsinger/bin/tex2ps my\_tex\_file another\_tex joe
and should get:
my\_tex\_file.ps another\_tex.ps joe.ps
with the original files still there, no intermediary files, and only
my own personal signature output. All the rest of the junk should be gone.

Go me!

\section*{Questions? Problems?}

E-mail arsinger@cs.brown.edu

While I certainly don't know everything about LaTeX, I'm learning and may be
able to help out.

\end{document}
